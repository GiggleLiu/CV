\documentclass[a4paper]{letter}
\usepackage{amssymb}
\usepackage{bbm}
\usepackage[normalem]{ulem}
\usepackage{graphicx}
\usepackage[colorlinks,linkcolor=blue,anchorcolor=blue,citecolor=blue,urlcolor=blue]{hyperref}
\usepackage{mciteplus}
\usepackage{etoolbox}
\usepackage{subcaption}
\usepackage{tikz}
\usetikzlibrary{shapes}
\usepackage{fontspec}

\title{Cover letter}
\author{JinGuo Liu\\ Department of Physics, Harvard University}
\date{\today}

\setlength{\topmargin}{-10mm}
\setlength{\textwidth}{7in}
\setlength{\oddsidemargin}{-8mm}
\setlength{\textheight}{9in}
\setlength{\footskip}{1in}

\newfontfamily\DejaSans{DejaVu Sans}

\begin{document}
\fontsize{10}{13}
\selectfont

Dear Prof. Pan Zhang,


I am writing to apply for the position of Assistant Professor at the Institute of Theoretical Physics Chinese Academy of Sciences (ITP, CAS), as advertised on your institute's WeChat platform.
I am currently a Post-Doctoral fellow at Harvard University and am expected to go back to China in July.
I am extremely interested in obtaining an assistant researcher position at the ITP, CAS, where I can contribute to its focus on computational complexity, quantum many-body methods, and quantum computing.
I got my Ph.D. degree in theoretical physics advised by Qiang-Hua Wang at Nanjing University. 
I also have two Postdoc experiences: one is at Lei Wang's group at Institute of Physics, Chinese Academy of Sciences, and the other is at Mikhail Lukin's group at Harvard University.
Dual background in computer science (bachelor) and physics gave me a deeper insight into the connection between computer science and physics.

My career in the recent five years was deeply influenced by the words from Lei: ``Working in the right direction is much more important than picking low-hanging fruits.''.
Compared to five years ago, now I have a better understanding about what is the right direction to work in. I am no longer satisfied by doing incremental works and start to challenge problems that are intrinsically hard and think about problems that people never thought about.
For example, in one of my unpublished papers in the attached representative papers, I proposed a new problem of solution space property computation.
Solving it with the generic tensor network method that I initially proposed provides many insights to another experiment paper about solving the maximum independent set problem by embedding it to Rydberg atom arrays.
The experiment paper was submitted to Science, and one of the referees highly praised the numeric method we used to study the problem.
About challenging hard problems, I and my collaborators solved the problem of optimal mapping from the maximum-independent-set problem on general graphs to the maximum-independent-set problem on unit disk grid graphs.
This reduction decreased the mapping overhead from $n^8$ to $n^2$ and enabled us to solve a general maximum-independent-set problem with Rydberg atom arrays.

I would enjoy discussing this position with you in the weeks to come. In the meantime, if you require any additional materials or information, I am happy to supply them. Thank you very much for your consideration.

Sincerely, Jinguo Liu

\end{document}
