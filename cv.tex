% !TEX program = xelatex
% !TEX encoding = UTF-8 Unicode
% Awesome CV LaTeX Template for CV/Resume
%
% This template has been downloaded from:
% https://github.com/posquit0/Awesome-CV
%
% Author:
% Claud D. Park <posquit0.bj@gmail.com>
% http://www.posquit0.com
%
% Template license:
% CC BY-SA 4.0 (https://creativecommons.org/licenses/by-sa/4.0/)
%


%-------------------------------------------------------------------------------
% CONFIGURATIONS
%-------------------------------------------------------------------------------
% A4 paper size by default, use 'letterpaper' for US letter
\documentclass[11pt, a4paper]{awesome-cv}
\usepackage{xeCJK}

% Configure page margins with geometry
\geometry{left=1.4cm, top=.8cm, right=1.4cm, bottom=1.8cm, footskip=.5cm}

% Specify the location of the included fonts
\fontdir[fonts/]

% Color for highlights
% Awesome Colors: awesome-emerald, awesome-skyblue, awesome-red, awesome-pink, awesome-orange
%                 awesome-nephritis, awesome-concrete, awesome-darknight
\colorlet{awesome}{awesome-red}
% Uncomment if you would like to specify your own color
% \definecolor{awesome}{HTML}{CA63A8}

% Colors for text
% Uncomment if you would like to specify your own color
% \definecolor{darktext}{HTML}{414141}
% \definecolor{text}{HTML}{333333}
% \definecolor{graytext}{HTML}{5D5D5D}
% \definecolor{lighttext}{HTML}{999999}

% Set false if you don't want to highlight section with awesome color
\setbool{acvSectionColorHighlight}{true}

% If you would like to change the social information separator from a pipe (|) to something else
\renewcommand{\acvHeaderSocialSep}{\quad\textbar\quad}


%-------------------------------------------------------------------------------
%	PERSONAL INFORMATION
%	Comment any of the lines below if they are not required
%-------------------------------------------------------------------------------
% Available options: circle|rectangle,edge/noedge,left/right
% \photo{./examples/profile.png}
\name{Jinguo Liu}{\huge (刘金国)}
\position{Postdoc fellow in Harvard university}
%\address{31 Miller St.}

\mobile{(+86) 1519-5955-770}
\email{cacate0129@gmail.com}
\homepage{https://giggleliu.github.io/}
\github{GiggleLiu}
% \gitlab{gitlab-id}
% \stackoverflow{SO-id}{SO-name}
% \twitter{@twit}
% \skype{skype-id}
% \reddit{reddit-id}
% \medium{madium-id}
 \googlescholar{4edw228AAAAJ}{Jinguo Liu}
%% \firstname and \lastname will be used
% \googlescholar{googlescholar-id}{}
% \extrainfo{extra informations}

\quote{``朝正确的方向攀爬,而不是去摘下垂的果实。''}

%-------------------------------------------------------------------------------
\begin{document}

% Print the header with above personal informations
% Give optional argument to change alignment(C: center, L: left, R: right)
\makecvheader

% Print the footer with 3 arguments(<left>, <center>, <right>)
% Leave any of these blank if they are not needed
\makecvfooter
  {\today}
  {Jinguo Liu (刘金国)~~~·~~~Curriculum Vitae}
  {\thepage}


%-------------------------------------------------------------------------------
%	CV/RESUME CONTENT
%	Each section is imported separately, open each file in turn to modify content
%-------------------------------------------------------------------------------
\cvsection{Education}
\begin{cventries}
  \cventry
    {B.S. in Software Engineering}
    {Nanjing Institute of Technology}
    {Nanjing}
    {2008--2012}
    {\begin{cvitems}
     When I was a college student, I read a book "Quantum Computation and Quantum Information" by Michael A. Nielsen.
        I was deeply impressed by the beautiful computation framework in the book, and decided to learn more about quantum computing in Prof. Yang Yu's group in Nanjing University.
    \end{cvitems}}
  \cventry
    {Ph.D. Theoretical Physics}
    {Nanjing University}
    {Nanjing}
    {2012--2017}
    {\begin{cvitems}
    Adviced under Prof. Qianghua Wang, doing numeric simulation of condensed matters.
    I mastered tensor networks algorithms and renormalizationing group theories, and became a geek in simulating quantum many body systems. Most of my works are about designing numeric algorithms to solve important problems in physics, like multi-channel Kondo problem and fractional topological excitation.
    In the last year as a doctor candidate, I won the first prize in ZTE fantastic algorithm challenge, which is a good proof of my solid algorithmic background of matrix computation and combinatorial optimization.
    \end{cvitems}}
\end{cventries}

\cvsection{Skills}
\begin{cvskills}
    \cvskill{Programming}{Julia, Python, Fortran}
    \cvskill{Language}{Chinese, English}
    \cvskill{Algorithms}{Tensor Networks, Differentiable Programming}
    \cvskill{Knowledge}{Quantum computing, Condensed matter physics, Combinatorial optimization}
\end{cvskills}
\cvsection{Experience}
\begin{cventries}
  \cventry
    {Postdoc}
    {Institute of Physics (IOP), Chinese Academy of Sciences}
    {Beijing}
    {2017--2019}
    {\begin{cvitems}
        Then I became a postdoc of a young and charming guy \href{http://wangleiphy.github.io/}{Lei-Wang}. Besides providing valuable suggestions in my research, Lei also creates a lot of opportunities for me, like encouraging me to give lectures and talks in international conferences and summer schools.
        My postdoc career is in Institute of Physics (IOP), Chinese Academy of Sciences. That time my research interest is automatic differentiation and quantum algorithms, this is a field that can incubate several killer Apps. I also developed the quantum simulation framework \href{https://github.com/QuantumBFS/Yao.jl}{Yao.jl} with a built in automatic differention engine together with a genuine Julia lover \href{http://blog.rogerluo.me/}{Xiu-Zhe Luo}.
        I mentored a student for Julia on the \href{https://summerofcode.withgoogle.com/}{GSoC} project of differentiable tensor networks. It is a valuable experimence for me to lead a project. The open source repository \texttt{OMEinsum} is listed bellow.
    \end{cvitems}}
  \cventry
    {Consultant}
    {QuEra computing}
    {Waterloo}
    {2020.01-2020.07}  % description
    {}
  \cventry
    {Postdoc fellow}
    {Harvard university}
    {Boston}
    {2020.08-}  % description
    {}
\end{cventries}
% I am a computational quantum physicsist. Armed with solid background of both quantum physics and computer science, I am able to solve some valuable problems in the cross discipline of quantum physics and computer sciences.
% I am also the maintainer of several open source projects (listed at the end of this CV), as well as an organizer of \href{https://github.com/QuantumBFS/}{QuantumBFS}.
% I list my research experiences as the following:
% \begin{itemize}
%     \item [6] Now I am full participated in reversible Turing machine. It can solve the most important problem in differential programming, the genuine automatic differentiation.
% \end{itemize}
%\input{cv/extracurricular.tex}
\cvsection{Honors \& Awards}
\begin{cvhonors}
    \cvhonor
    {First prize (out of 8000 teams, 100,000 RMB award)} % Award
    {\href{http://www.iqiyi.com/w\_19rto3v4h1.html}{ZTE Fantastic Algorithm Challenge}} % Event
    {Xi An, China} % Location
    {2017} % Date(s)

    \cvhonor
    {First prize}{Physics Olympiad}{JiangSu Province, China}{2007}
    \cvhonor
    {Academic Excellence Scholarship}{Nanjing University}{NanJing}{2016}
\end{cvhonors}

%\input{cv/presentation.tex}
\cvsection{Selected Presentations}
\begin{cventries}
\cventry
    {Presenter} % role
    {March Meeting} % event
    {Boston} % location
    {2019} % date
    {
    \begin{cvitems}
        Gave a talk \href{https://meetings.aps.org/Meeting/MAR19/Session/E27.10}{"Differentiale Quantum Circuits and Generative Modeling"}
    \end{cvitems}
    }
\cventry
    {Presenter}
    {Juliacon}{Baltimore}{2019}
    {
    \begin{cvitems}
        Gave a talk \href{https://www.youtube.com/watch?v=f-CaQMTqjPk}{"Differential Programming Tensor Networks"}
    \end{cvitems}
    }
\cventry
    {Lecturer}
    {\href{https://github.com/QuantumBFS/SSSS}{Deep Learning and Quantum Programming: A Spring School}}
    {Dongguan}{2019}
    {
    \begin{cvitems}
        Gave lectures on quantum computing.
    \end{cvitems}
    }

\end{cventries}

%\input{cv/writing.tex}
%\input{cv/committees.tex}
\cvsection{Selected Publications}
\begin{cventries}
  \cventry
    {First author} % Role
    {Computing properties of independent sets by generic programming tensor networks} % Title
    {Unpublished} % Location
    {2022} % Date(s)
    {
      \begin{cvitems} % Description(s)
        \item {}
      \end{cvitems}
    }
  \cventry{First author}
  {Tropical tensor network for ground states of spin glasses}
  {Phys. Rev. Lett. 126, 090506}
  {2021}
  {
      \begin{cvitems} % Description(s)
        \item {}
      \end{cvitems}
  }
  \cventry
    {Second author} % Role
    {Yao.jl: Extensible, Efficient Framework for Quantum Algorithm Design} % Title
    {Quantum} % Location
    {2020} % Date(s)
    {
      \begin{cvitems} % Description(s)
        \item {One of the main authors of the most popular quantum circuit simulator in Julia language.}
      \end{cvitems}
    }
\end{cventries}
%-------------------------------------------------------------------------------
\end{document}
