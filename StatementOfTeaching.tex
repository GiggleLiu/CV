\documentclass[a4paper]{article}
\usepackage{amssymb}
\usepackage{bbm}
\usepackage[normalem]{ulem}
\usepackage{graphicx}
\usepackage[colorlinks,linkcolor=blue,anchorcolor=blue,citecolor=blue,urlcolor=blue]{hyperref}
\usepackage{mciteplus}
\usepackage{etoolbox}
\usepackage{subcaption}
\usepackage{tikz}
\usetikzlibrary{shapes}
\usepackage{fontspec}

\title{Statement of Teaching}
\author{JinGuo Liu\\ Department of Physics, Harvard University}
\date{\today}

\newcommand{\<}{\langle}
\renewcommand{\>}{\rangle}
\newcommand{\vsigma}{\vec{\sigma}}
\setlength{\topmargin}{-10mm}
\setlength{\textwidth}{7in}
\setlength{\oddsidemargin}{-8mm}
\setlength{\textheight}{9in}
\setlength{\footskip}{1in}

\newfontfamily\DejaSans{DejaVu Sans}

\begin{document}
\fontsize{10}{13}
\selectfont
\maketitle

I have two short teaching experiences that require me to think about what is and how to teach.
One is in the \href{https://for1807.physik.uni-wuerzburg.de/2017/07/11/winter-school-2018/}{The FOR 1807 Winter School on Numerical Methods for Strongly Correlated Quantum Systems}, I and Shuo-Hui Li prepared one of the hands-on sessions.
The other is in the \href{https://github.com/QuantumBFS/SSSS}{Song-Shan-Lake Spring School}, I prepared the lectures on the quantum computing part as one of the organizers.
%Both are happy and enjoyable experiences thanks to my advisor Lei's patient guide.
In both schools, we get positive feedback. In the Marburg winter school, Prof. Fakher Assaad commented to his student in front of Lei: ``See, this is professional.''.
In the Song-Shan Lake Spring School, we also get a high score in the students' feedback survey.

In the Marburg hands-on lecture, I and my collaborators first introduced briefly automatic differentiation and high-performance computing.
Then we show some examples. After showing an example, we gave students 10 minutes to play with the online notebook, which is directly accessible with a browser without technical barriers.
In the Song-Shan Lake Spring school, I can have more control over the overall teaching program.
Besides giving lectures, I and other lecturers prepared the ``happy fatty night''.
We prepared free food for students at around 8 PM to attract students to gather and ask questions.
These questions range from setting up the development environment to future research directions.
From our previous experience, many students got stuck in the stage of the environment set up in a numeric course, so it is good to give them a bootstrap.
For those having a solid numeric background, we prepared a \href{https://github.com/QuantumBFS/SSSS/blob/master/Challenge.md}{challenge} for them.
The student submitting the most solutions can get a MacBook as a reward.
All of us are surprised by the diverse solutions that students submitted to Github: \href{https://github.com/quantumbfs0?tab=stars}{https://github.com/quantumbfs0?tab=stars}.
We learned a lot from students in turn.

Here are the bullet points about what I learned from these two teaching experiences
\begin{itemize}
    \item It is beneficial to set up a goal for teaching at the early stage of lecture preparation. At the beginning of each lecture, it is helpful to let the students know what they can get from this lecture (or the goal), and these points should be emphasized repeatedly in the rest of the lecture.
    \item During the teaching, it is crucial to control the speed.
It is better to stop at a certain point and get feedback from students than to go ``smoothly'' without stopping.
Because when people encounter a new concept, he needs to think for quite a while in their head before accepting it. It is also good to have at least one dynamic-timed sessions that take any time you want. A teaching practice does not always work as planned. Then a dynamic-timed session will help one manage the time.
    \item It is important to prepare and archive the teaching material online so that students can access them after the course. Github is a good form to organize materials.
    \begin{itemize}
        \item Materials for the hands-on session in the winter school in Marburg: \href{https://github.com/GiggleLiu/marburg}{https://github.com/GiggleLiu/marburg}
        \item Materials for the Song-Shan-Lake spring school \href{https://github.com/QuantumBFS/SSSS}{https://github.com/QuantumBFS/SSSS}.
    \end{itemize}
    \item For students having diverse backgrounds, it is helpful to prepare some challenges targeting students with a solid background and ``happy fatty night'' (or a time for students to socialize and ask ``dumb'' questions) targeting students with a less solid background. Try to make sure most students can learn something from the lectures.
\end{itemize}

\section{Teaching plan}
I enjoy teaching. I am primarily interested in teaching algorithms, which can be in the form of 
\begin{itemize}
    \item a language course like Julia programming
    \item numeric methods in physics
    \item an introductory course about combinatorial optimization and computational hardness
\end{itemize}
If I can teach any of the above courses, I can do better than most lecturers. I am also happy to teach quantum computing basics or general theoretical computing models.

\end{document}
