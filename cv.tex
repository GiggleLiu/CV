% !TEX program = xelatex
% !TEX encoding = UTF-8 Unicode
% Awesome CV LaTeX Template for CV/Resume
%
% This template has been downloaded from:
% https://github.com/posquit0/Awesome-CV
%
% Author:
% Claud D. Park <posquit0.bj@gmail.com>
% http://www.posquit0.com
%
% Template license:
% CC BY-SA 4.0 (https://creativecommons.org/licenses/by-sa/4.0/)
%


%-------------------------------------------------------------------------------
% CONFIGURATIONS
%-------------------------------------------------------------------------------
% A4 paper size by default, use 'letterpaper' for US letter
\documentclass[11pt, a4paper]{awesome-cv}
\usepackage{xeCJK}
\hypersetup{
    colorlinks = true,
}

% Configure page margins with geometry
\geometry{left=1.4cm, top=.8cm, right=1.4cm, bottom=1.8cm, footskip=.5cm}

% Specify the location of the included fonts
\fontdir[fonts/]

% Color for highlights
% Awesome Colors: awesome-emerald, awesome-skyblue, awesome-red, awesome-pink, awesome-orange
%                 awesome-nephritis, awesome-concrete, awesome-darknight
\colorlet{awesome}{awesome-red}
% Uncomment if you would like to specify your own color
% \definecolor{awesome}{HTML}{CA63A8}

% Colors for text
% Uncomment if you would like to specify your own color
% \definecolor{darktext}{HTML}{414141}
% \definecolor{text}{HTML}{333333}
% \definecolor{graytext}{HTML}{5D5D5D}
% \definecolor{lighttext}{HTML}{999999}

% Set false if you don't want to highlight section with awesome color
\setbool{acvSectionColorHighlight}{true}

% If you would like to change the social information separator from a pipe (|) to something else
\renewcommand{\acvHeaderSocialSep}{\quad\textbar\quad}


%-------------------------------------------------------------------------------
%	PERSONAL INFORMATION
%	Comment any of the lines below if they are not required
%-------------------------------------------------------------------------------
% Available options: circle|rectangle,edge/noedge,left/right
% \photo{./examples/profile.png}
\name{Jinguo Liu}{\huge (刘金国)}
\position{Postdoc fellow in Harvard university}
%\address{31 Miller St.}

\mobile{(+86) 1519-5955-770}
\email{cacate0129@gmail.com}
\homepage{https://giggleliu.github.io/}
\github{GiggleLiu}
% \gitlab{gitlab-id}
% \stackoverflow{SO-id}{SO-name}
% \twitter{@twit}
% \skype{skype-id}
% \reddit{reddit-id}
% \medium{madium-id}
 \googlescholar{4edw228AAAAJ}{Jinguo Liu}
%% \firstname and \lastname will be used
% \googlescholar{googlescholar-id}{}
% \extrainfo{extra informations}

\quote{``朝正确的方向努力,而不是去摘下垂的果实。''}

%-------------------------------------------------------------------------------
\begin{document}

% Print the header with above personal informations
% Give optional argument to change alignment(C: center, L: left, R: right)
\makecvheader

% Print the footer with 3 arguments(<left>, <center>, <right>)
% Leave any of these blank if they are not needed
\makecvfooter
  {\today}
  {Jinguo Liu (刘金国)~~~·~~~Curriculum Vitae}
  {\thepage}


%-------------------------------------------------------------------------------
%	CV/RESUME CONTENT
%	Each section is imported separately, open each file in turn to modify content
%-------------------------------------------------------------------------------
\cvsection{Education}
\begin{cventries}
  \cventry
    {B.S. in Software Engineering}
    {Nanjing Institute of Technology}
    {Nanjing}
    {2008--2012}
    {\begin{cvitems}
     I was a pioneer of the open-source software movement in my institute. Deeply impressed by the beautiful computation framework in the book "Quantum Computation and Quantum Information" by Michael A. Nielsen, I was eager to learn more about quantum computing.
    \end{cvitems}}
  \cventry
    {Ph.D. Theoretical Physics}
    {Nanjing University}
    {Nanjing}
    {2012--2017}
    {\begin{cvitems}
    Advised under Prof. Qianghua Wang, I built up my interest in algorithms for solving quantum many-body systems.
    I mastered tensor networks algorithms and renormalization group theories and became a geek in simulating quantum many-body systems. Most of my works are about designing new algorithms to solve problems in physics, like the multi-channel Kondo problem and fractional topological excitation.
    In the last year as a doctoral candidate, I won the first prize in the ZTE fantastic algorithm challenge, which reflects my solid algorithmic background in matrix computation and combinatorial optimization.
    \end{cvitems}}
\end{cventries}

\cvsection{Skills}
\begin{cvskills}
    \cvskill{Programming}{Julia, Python, Fortran}
    \cvskill{Language}{Chinese, English}
    \cvskill{Knowledge}{Tensor Networks, Differential Programming, Quantum computing, Computational complexity, Condensed matter physics, Combinatorial optimization, High performance computing}
\end{cvskills}
\cvsection{Experience}
\begin{cventries}
  \cventry
    {Postdoc}
    {Institute of Physics (IOP), Chinese Academy of Sciences (CAS)}
    {Beijing}
    {2017--2019}
    {
      I became a postdoc in \href{http://wangleiphy.github.io/}{Lei-Wang}'s group, one of the smartest people I knew. Besides providing valuable advice about research, Lei also provides opportunities for me to give lectures and talks at international conferences and summer schools. At that time, my research interest is automatic differentiation and quantum algorithms.
    }

  \cventry
    {Consultant}
    {QuEra Computing Inc.}
    {Waterloo}
    {2020.01-2020.07}  % description
    {
        Due to the COVID, I was trapped in Waterloo - a wild place where you can see wild animals on the streets. QuEra kindly offered me a full-time consultant job. I worked on stochastic optimizers for variational quantum algorithms and classical benchmarking quantum approximation optimization algorithm (QAOA).    }

  \cventry
    {Postdoc}
    {Harvard university}
    {Boston}
    {2020.08-}  % description
    {
    QuEra also sponsored my Postdoc in Mikhail Lukin's group.
    Working at Harvard is a unique experience for me. While my skills helped experimentalists and theorists in Misha's group, I learned more exciting stuff from people around me every day.
    \vspace{2em}
\begin{cvitems}
\item I developed generic tensor networks (tensor networks with generic element types) to understand the solution space properties of the maximum independent set problem. I learned their approach to analyzing hardness from the solution space geometry: the overlap gap property and adiabatic gap analysis.
\item I mapped the maximum independent set problem on a general graph to the one with restricted geometry of diagonal-coupled unit-disk grid graph that Rydberg atom arrays can implement (has been patented). I learned how to reduce many other hard problems to the maximum independent set problem.
\item I improved SLM hologram computation for generating arbitrary optical traps (will be patented). I learned how Fourier optics plays a role in the Rydberg atom experiment works in turn.
    \end{cvitems}}
\end{cventries}
% I am a computational quantum physicsist. Armed with solid background of both quantum physics and computer science, I am able to solve some valuable problems in the cross discipline of quantum physics and computer sciences.
% I am also the maintainer of several open source projects (listed at the end of this CV), as well as an organizer of \href{https://github.com/QuantumBFS/}{QuantumBFS}.
% I list my research experiences as the following:
% \begin{itemize}
%     \item [6] Now I am full participated in reversible Turing machine. It can solve the most important problem in differential programming, the genuine automatic differentiation.
% \end{itemize}
%\input{cv/extracurricular.tex}
\cvsection{Honors \& Awards}
\begin{cvhonors}
    \cvhonor
    {First prize}{Physics Olympiad}{JiangSu, China}{2007}
    \cvhonor
    {Academic Excellence Scholarship}{Nanjing University}{NanJing}{2016}
    \cvhonor
    {First prize (out of 8000 teams, 100,000 RMB award)} % Award
    {\href{http://www.iqiyi.com/w\_19rto3v4h1.html}{ZTE Fantastic Algorithm Challenge}} % Event
    {Xi An, China} % Location
    {2017} % Date(s)
\end{cvhonors}

\cvsection{Open Source Contributions}
\begin{cventries}
  \cventry
    {One of the main developers} % Role
    {\href{https://github.com/QuantumBFS/Yao.jl}{Yao.jl}} % Title
    {} % Location
    {} % Date(s)
    {
      \begin{cvitems} % Description(s)
          {\texttt{Yao.jl} is the most popular quantum circuit simulation framework in the Julia community.
          The Yao repository has 650+ Github stars, and the paper has 50+ citations.
          It is fast, generic, GPU accelerated, and differentiable.
          }
      \end{cvitems}
    }
  \cventry
    {Mentor of \texttt{OMEinsum.jl}, main developer of \texttt{OMEinsumContractionOrders.jl}} % Role
    {\href{https://github.com/under-Peter/OMEinsum.jl}{OMEinsum.jl} and \href{OMEinsumContractionOrders.jl}{OMEinsumContractionOrders.jl}} % Title
    {}{}
    {
      \begin{cvitems} % Description(s)
          {\texttt{OMEinsum.jl} is a generic, differentiable einsum library with GPU support. It was developed by Andreas Peter (mentor under me) on the \href{https://summerofcode.withgoogle.com/}{Google Summer of Code (GSoC)} project about differential programming tensor networks. This project is a successful one and now its Github repo has 100+ stars.
          \texttt{OMEinsumContractionOrders.jl} is its extension for contraction order optimization that many state-of-the-art algorithms implemented in it.}
      \end{cvitems}
    }

  \cventry
    {Main developer}
    {\href{https://github.com/Happy-Diode/GraphTensorNetworks.jl}{GraphTensorNetworks.jl}} % Title
    {}{}
    {
      \begin{cvitems} % Description(s)
          {\texttt{GraphTensorNetworks.jl} is a package using generic tensor network contraction for solving graph properties. It comes together with the paper: \textit{``Computing solution space properties by generic programming tensor networks''} (see section ``Selected Publications'').}
      \end{cvitems}
    }

\end{cventries}

%\input{cv/presentation.tex}
\cvsection{Selected Presentations}
\begin{cventries}
    \cventry
    {Lecturer}
    {The FOR 1807 Winter School on Numerical Methods for Strongly Correlated Quantum Systems}
    {Marburg}
    {2018}
    {
    \begin{cvitems}
        Lecture: \href{https://github.com/GiggleLiu/marburg}{"Deep learning and quantum many body systems"}.
    \end{cvitems}
    }
\cventry
    {Speaker} % role
    {March Meeting} % event
    {Boston} % location
    {2019} % date
    {
    \begin{cvitems}
        Talk: \href{https://meetings.aps.org/Meeting/MAR19/Session/E27.10}{"Differentiale Quantum Circuits and Generative Modeling"}
    \end{cvitems}
    }
\cventry
    {Speaker}
    {Juliacon}{Baltimore}{2019}
    {
    \begin{cvitems}
        Talk: \href{https://www.youtube.com/watch?v=f-CaQMTqjPk}{"Differential Programming Tensor Networks"}
    \end{cvitems}
    }
\cventry
    {Lecturer}
    {\href{https://github.com/QuantumBFS/SSSS}{Deep Learning and Quantum Programming: A Spring School}}
    {Dongguan}{2019}
    {
    \begin{cvitems}
        Lectures on the quantum computing part, one of the main organizers of ``happy fatty night'' (coding party for resolving student's issues).
    \end{cvitems}
    }

\end{cventries}

%\input{cv/writing.tex}
%\input{cv/committees.tex}
\cvsection{Selected Publications}
\begin{cventries}
  \cventry
    {Jinguo Liu, Min-Thi Nguyen, Shengtao Wang et al and Hannes Pichler} % Role
    {Maximum independent sets: from unit disk graphs to arbitrary connectivity} % Title
    {Unpublished} % Location
    {2022} % Date(s)
    {
      \begin{cvitems} % Description(s)
        {
        Recent progress in variational quantum algorithms shows the potential of using Rydberg atom arrays to solve the maximum independent set (MIS) problem defined on diagonal-coupled unit-disk grid graphs.
    We propose a scheme to reduce the problem of finding an MIS of a general graph $G$ to these graphs with highly restricted geometry to boost the power of these quantum algorithms.
          The mapped graph has an overhead bounded by $4 {\rm pw}(G)$, where ${\rm pw}(G)$ is the path width of graph $G$.
    We show the proposed mapping scheme is optimal up to a constant if there are no sub-exponential algorithms for finding an MIS of a general graph.
    Understanding its value to quantum algorithms, we patented it.
}
      \end{cvitems}
    }
  \cventry
    {Jinguo Liu, Xun Gao, Shengtao Wang, Midelyn Cain and Mikhail Lukin} % Role
    {Computing solution space properties by generic programming tensor networks} % Title
    {Unpublished} % Location
    {2022} % Date(s)
    {
      \begin{cvitems} % Description(s)
      {We introduce a tensor network algorithm %a method that combines generic programming and tensor network techniques
to compute various solution space properties of a class of combinatorial optimization problems on graphs that can be rephrased as satisfiability problems of constraints over local sets of vertices, including the independent set problem, the maximum cut problem, the vertex coloring problem, the maximal clique problem, the dominating set problem, and the satisfiability problem, among others.
We look at the independent set problem as an example, and show how to compute the size of the maximum independent set, count the number of independent sets of a given size, and enumerate/sample the independent sets of a given size.
By using generic programming techniques, %which require a program to work correctly for different input data types without sacrificing efficiency,
the same simple-to-implement framework can be used to compute all of these properties. Our algorithm utilizes recent advances in tensor network contraction techniques to achieve high performance, including methods to quickly find a near-optimal contraction order and slicing.
          To demonstrate how our versatile tool helps to understand these hard problems, we apply it to a few examples, including computing the entropy constant for several hardcore gases on disordered lattices, studying the Overlap Gap Property on unit disk graphs and regular graphs, and analyzing the performance of quantum and classical optimization algorithms for the independent set problem.}
      \end{cvitems}
    }
  \cventry{Jinguo Liu, Lei Wang and Pan Zhang}
  {Tropical tensor network for ground states of spin glasses}
  {Phys. Rev. Lett. 126, 090506}
  {2021}
  {
  }
  \cventry
    {Xiuzhe Luo, Jinguo Liu, Pan Zhang and Lei Wang} % Role
    {Yao.jl: Extensible, Efficient Framework for Quantum Algorithm Design} % Title
    {Quantum} % Location
    {2020} % Date(s)
    {
    }
\end{cventries}
%-------------------------------------------------------------------------------
\end{document}
