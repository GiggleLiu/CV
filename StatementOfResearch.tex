\documentclass[a4paper]{article}
\usepackage{amssymb}
\usepackage{bbm}
\usepackage[normalem]{ulem}
\usepackage{graphicx}
\usepackage[colorlinks,linkcolor=blue,anchorcolor=blue,citecolor=blue,urlcolor=blue]{hyperref}
\usepackage[numbers]{natbib}
\usepackage{mciteplus}
\usepackage{etoolbox}
\patchcmd{\thebibliography}{\section*{\refname}}{}{}{}
\title{Statement of Research}
\author{JinGuo Liu}
\date{\today}

\newcommand{\<}{\langle}
\renewcommand{\>}{\rangle}
\newcommand{\vsigma}{\vec{\sigma}}
\setlength{\topmargin}{-10mm}
\setlength{\textwidth}{7in}
\setlength{\oddsidemargin}{-8mm}
\setlength{\textheight}{9in}
\setlength{\footskip}{1in}

\begin{document}
\fontsize{10}{13}
\selectfont
\maketitle

\section{Overview of Research Interests}
I have broad interests in numerical simulations of strongly correlated quantum systems in condensed matter physics.
Problems that I have covered are electronic instability (especially superconductivity)\cite{Liu2014}, impurity problem\cite{Liu2016a} and topological states of matter\cite{Liu2016b}.
My current interest lies in many body localization(MBL), thermalization and quantum entanglement.
%I hope to create a fresh view for both thermalized and localized systems from theory of measurements.
%My current plan is to devise a Hamiltonian with spin degree of freedom localized by disorder while charge degree of freedom thermal.
One of my most ambitious projects is to create a valid approach to optimize tensor network with state of art deep learning lagorithms, in order to study thermal systems.(Sec.\ref{proposal})

Comparing with most doctor candidates, I am more familiar with numerical algorithms like optimization, matrix computation and machine learning et al.
I have got my bachelor degree on Computer Science. I have not only rich coding experience in programming languages like C++, Java, Python and Fortran,
but also know a lot on object oriented prototyping, data structure, database, web scraping, building websites and parallel computation et al.
With respect to numerical simulation in condensed matter physics,
I am experienced in using large sparse matrices, tensor networks and Monte Carlo.
Developing new approaches to modeling and solving physical problems is also one of my major interests.

For two works with myself the first author, it's fair to say both original ideas came from myself.
The first focuses on impurity problems, which is about how an quantum impurity with interaction behave when coupled to a non-interacting continuous bath(that can be represented by a hybridization function).
What I have done is devising a general scheme of mapping a multi-channel bath into a chain and using it to study complex impurities in superconducting baths.
The other work focuses on non-trivial excitations relating two states.
I propose to use sub-system fidelity(SSF) defined by comparing a pair of reduced density matrices to pinpoint the location of excitations.
We used this scheme to study "Bell entangled States" in systems with symmetry protect topological(SPT) order, and cat states in symmetry breaking systems.
The calculation of SSF is based on the framework of matrix product states(MPS).
The following section describes my researches in more detail.

%I have tried multiple numerical methods like functional renormalization group(FRG),
%numerical renormalization group(NRG), dynamic mean field theory(DMFT), Quantum Monte Carlo(QMC) and density matrix renormalization group(DMRG) et al,
%most of them, I have my own version of code in python(with fortran boost).
%I like trying new stuffs, and am able to think of new ideas, as a proof, initial ideas of two of my works came out of my mind.
%Monte Carlo is a classical method to study strongly correlated systems without serious sign problem.
%Numerical strategies related to tensor networks are developing fast recent decades,
%it involves NRG(1975) in 0-dimension, DMRG(1993) in low dimension,
%and iTEBD(2003), TRG(2007), SRG(2010), HOTRG(2012), TNR\&loop-TNR(2015) for higher dimensions.
%Besides, some other fields like big data, neural network also benefit a lot from tensor network representation.

\section{Previous Work}
\subsection{Quantum impurity in channel mixing baths}
Solving an impurity problem is important for not only itself, but also for its broad uses in DMFT.
Widely used numerical impurity solvers include continuous time quantum Monte Carlo(CTQMC) and numerical renormalization group(NRG), exact diagonalization(ED) and so on.
NRG scheme was devised by Wilson. Comparing with CTQMC, NRG is sign problem free and work in real frequency directly but hard to be generalized to multi-channel problems.
\begin{figure}
    \begin{center}
    \includegraphics[width=8.5cm]{multi-band.eps}
    \end{center}
\caption{(a) Illustration of a Wilson chain mapped from a two-channel bath without channel mixing, the dashed lines are non-zero hopping terms.
(b) The same chain mapped from a general channel mixing bath.}\label{wilsonchain}
\end{figure}

%Being a two step scheme, it first maps a continuous hybridization function to a discrete real space Wilson chain model and then use NRG iterations to solve this chain.
As the notion of matrix product state(MPS) becomes popular,
new methods like variational matrix product state\cite{Weichselbaum2009}
make us see the hope to break the curse of exponential growth of bond dimension in NRG as number of channels grows.
%Besides, with full density matrix formalism based on MPS, finite temperature and dynamic properties of impurities can be easily extracted using NRG and VMPS.\cite{Merker2012,Nghiem2014}
However, there is still a gap towards solving complicated impurity problems, that is how to map a general hybridization function $\Delta(\omega)= i [\Sigma(\omega+i0^+)-\Sigma(\omega-i0^+) ]/2\pi$ to a Wilson chain required by both.
This is exactly the problem that I have solved.
I developed for the first time a scheme for mapping a general multi-channel bath to a Wilson chain,
which contains derivation of the formula mapping a general multi-channel bath to discrete sites that directly coupled to the impurity, with user specified scaling behaviour(logarithmic or adaptive in NRG),
and block Lanczos tridiagonalization procedure that transforms it into a channel mixing Wilson chain(Fig. \ref{wilsonchain}).
This method can be exact in the infinite site limit.
The idea is simple, but in practise, in order to get the best performance, I used techniques like spline curve fitting, Runge-Kutta integration, high precision and iterative Gram-Schmidt algorithm, some of which are directly related to the reliability of the final result.
%Before my proposal, only non-channel mixing baths(or baths that can be decoupled directly) can be mapped to a real space chain.
In the case study, I analysed two kinds of complicated superconducting baths.
The first is an s-wave superconducting bath with particle-hole non-symmetric normal band.
My NRG result shows the region with Yu-Shiba-Rusinov(YSR) state is tilted due to particle-hole non-symmery as can be verified approximately through analytical approach.\cite{Meng2009}
The second case I studied is a d-wave superconducting bath but is extensively coupled to the impurity.
It was proved that the cooper pairs play no role in Kondo physics for impurity extensively coupled d-wave superconducting bath under some specific configurations.\cite{Fritz2005}
However we show that cooper pairs are able to screen a local momentum when the coupling is not $C_{4v}$ point group symmetric.
%With my proposal, Kondo physics in complicated systems like spin orbit coupled systems, multi-impurity problem and superconducting systems.

\subsection{Sub-system fidelity in topological phase of matter}

Degenerate states in different parity sectors in system with symmetry protect topological(SPT) order are known to be locally indistinguishable(LI) and could be related by zero modes at two edges.
However, for interacting systems, even though we have got two states using finite DMRG,
it is not easy to show the two states are related by edge modes or they are LI.%(in fact, Bell states and cat states are also good examples that LI).

I propose to use sub-system fidelity(SSF) $F={\rm Tr}\sqrt{\sqrt{\rho_1}\rho_2\sqrt{\rho_1}}$ to detect and characterize excitations in a subspace, especially when those excitations are highly non-local.
Here, $\rho_1$ and $\rho_2$ are reduced density matrices that obtained from the same region of two states.
SSF is a quantity that measures the minimal probability correlation that can be obtained from a set of positive operator valued measurements(POVM) on specified region.
Our work mainly answers the following two questions in an easy way in terms of measurements:
\begin{enumerate}
    \item When we get two states on a finite chain(suppose in the form of MPS), how can we verify they are locally indistinguishable(LI)?
        And what makes them LI?
    \item When we get degenerate states, how can we tell these states could be related by edge modes only?
\end{enumerate}
%SSF measures the fidelity(or similarity) between two reduced density matrices that obtained from same region of two states.
%(minimal probability correlation that can be obtained from a set of positive operator valued measurements(POVM)
%An intuitive thinking is that if we create a local excitation at area $i$ so that $O_i^\dag|\psi\>=|\psi'\>$ with $\<\psi|\psi'\>=0$,
%we would expect the SSF between these two states on this specific area $F^i_{\psi,\psi'}=0$, for we can easily distinguish these two states by measurements on this area.
%By some linear combinations of $|\psi\>$ and $|\psi'\>$, we can form many other orthogonal pairs on their Bloch sphere, all of them are distinguishable at area $i$.
Our work shows that to reveal the location and characteristic size of some special modes that are highly non-local(e.g. those relating SPT degenerate states and cat states) from two states,
it is not enough to compare(through SSF) two states only, but states on the Bloch sphere spanned by them.
I constructed the representation for edge modes and developed a criteria to examine whether these edge modes are only allowed modes in this subspace.
In practise, the coding involves two-site VMPS based on matrix product state(MPS) and matrix product operator(MPO) with U(1) good quantum number.
On the other hand, an algorithm for getting the mixed state for arbitrary segment, by tracing out the orthogonalized environment, also plays a vital role in SSF calculation.
In the implementation on the spin-1 Heisenberg model,
the degeneracy of entanglement spectrum\cite{Pollmann2010} is interpreted as the direct consequence of having "Bell entangled states" in this system.
Also, I used two length scales for edge modes to depict the transition from topological phase to $Z_2^z$ symmetry breaking phase.
Interacting Kitaev chain and XZ model(mathematically equivalent through Jordan-Wigner transformation) are studied in a comparative way,
in our scheme, it is too obvious why people are able to construct "Majorana edge modes"\cite{Vishveshwara2011,*Bardyn2012,*Zvyagin2013} in bosonic models mathematically,
while those modes have nothing to do with topology and two edges.
In fact, they can be defined anywhere in symmetry breaking systems.

\section{Future Directions}\label{proposal}
Recently, I became interested in many body localization(MBL) and eigen-state thermalization(ETH).
Numerical simulations play an important roles in the study of MBL such as
exact diagonalization(ED) and variational unitary matrix product operator (VUMPO)\cite{Pollmann2016} for energy statistics,
time evolving block decimation(TEBD) for time evolution and density matrix renormalization group(DMRG)-X\cite{Khemani2016} for targeting highly excited states.
I am familiar with most of these methods, also MBL is closely related to my previous works.
For example, SSF could be used to get the characteristic length(and construct the many body representation) combining DMRG-X scheme for an $l$-bit relating two states.
The characteristic length of $l$-bits as a function of energy is useful in probing the many body mobility edge.
However, my personal interest lies in thermal states.
A both important and challedging problem is how to get and represent a thermal state.

Machine learning became a trendy numerical method in many areas since the wave of research near 2006.
This research wave is driven by the concept of deep learning and the development of internet.
As an advanced version of optimization method, machine learning has been successfully implemented in big data processing.
Like quantum state representation, big data processing has also came to the era of tensor network\cite{Cichocki2014}.
This is the reason why I intended to combine machine learning and tensor network to solve physical problems.
In fact, machine learning has already been used in coping physical problems combining Monte Carlo simulations, by optimizing the Monte Carlo sampling procedure to speed up the program.\cite{Liu2016}
I believe tensor network can be a better playground for machine learning algorithms like structure revealing, back propagation et al.

To diagonalize a thermal Hamiltonian, we face difficulties in both state representation and optimization.
For example, in the VUMPO approach of getting full energy spectrum for a chain,
people face the problem of fast growing computational cost as the number of layers increases. Besides, if there would exist any good ansatz for a thermal states, it must be a multiple layered tensor network in order to meet the requirement of volume law entanglement entropy.
In order to generalize VUMPO to a thermal chain, we need a better optimization scheme other than variational local optimization to optimize the deep layers of tensor network.
In the following description, I intended to devise a circuit to get the inverse of an MPO through neural network training to justify the usefulness of neural network algorithms in tensor networks.

\begin{figure}
    \begin{center}
    \includegraphics[width=8.5cm]{plan.pdf}
    \end{center}
    \caption{Illustration of using machine learning to get the reverse of $H$ with a $4$-layered tensor network.
    White squares are our Hamiltonian and black squares are matrices for neurons' weights $X_i$.}\label{plan}
\end{figure}

As shown in Fig.\ref{plan}, I construct an $n$-layered MPO network $X=\prod\limits_{i=1}^n X_i$, with $X_i$ being initialized to a random MPO.
Training data are a large amount of random MPS $|\psi_0\>$ and the cost function is $J(X)=\||\psi_n\>-|\psi_0\>\|_2$ for we expect the overall effect of the network to be identity.
We train the network repeatedly until we have $|\psi_n\>=|\psi_0\>$ for any input, i.e. $X$ is the inverse of $H$.
Back propagation algorithm in machine learning can be implemented directly to perform updates for tensors,
with weights for change $|\delta X_j\>\propto |\delta \psi_j\>\<\psi_{j-1}|$.

Similar procedures should be possible for eigenvalue problems.\cite{Yu2015}
However, these proposals face many difficulties.
The most obvious one is finding an efficient scheme to evaluate tensor network in order to get output (especially for a thermal chain).
Sadly, tensor renormalization group algorithms can not be used directly here for we need states generated from applying MPO in each layer to train our 'hidden' units.
However, I believe, with try and error, a valid approach for solving MPO eigenvalue problems with machine learning will show up.

\section{Summary}

My dual background of computer science and physics and broad experience of numerical simulations of phenomena in condensed matter physics has built a solid ground for pursuing my academic career in physics.
I am very interested in working in your group and am confident that I will be helpful in especially numerical simulations.

%\bibliographystyle{ieeetr}
\bibliographystyle{rsc}
\bibliography{ref}
\bigskip
[14] A selection of my coding repositories on Github:
\begin{itemize}
    \item \href{https://github.com/GiggleLiu/Jacobi\_Davidson}{Jacobi-Davidson method for diagonalizing large sparse matrices}
    \item \href{https://github.com/GiggleLiu/nrg\_mapping}{Numerical Renormalization, the mapping}
    \item \href{https://github.com/GiggleLiu/pymps}{Matrix Product State and Matrix Product Operator}
\end{itemize}

\end{document}
