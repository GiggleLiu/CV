\documentclass[a4paper]{article}
\usepackage{amssymb}
\usepackage{bbm}
\usepackage[normalem]{ulem}
\usepackage{graphicx}
\usepackage[colorlinks,linkcolor=blue,anchorcolor=blue,citecolor=blue,urlcolor=blue]{hyperref}
\usepackage{mciteplus}
\usepackage{etoolbox}
\title{Statement of Research}
\author{JinGuo Liu\\ Department of Physics, Harvard University}
\date{\today}

\newcommand{\<}{\langle}
\renewcommand{\>}{\rangle}
\newcommand{\vsigma}{\vec{\sigma}}
\setlength{\topmargin}{-10mm}
\setlength{\textwidth}{7in}
\setlength{\oddsidemargin}{-8mm}
\setlength{\textheight}{9in}
\setlength{\footskip}{1in}

\begin{document}
\fontsize{10}{13}
\selectfont
\maketitle

I am a Post-Doctoral Fellow at the department of physics at Harvard University with an interest in understand the connection between computing and physics.
In the past decades, these two seemingly unrelated fields are more and more interweaved.
It brings many beautiful theories that deepened our understanding toward the nature of computing and the nature of our physical world.
For example, by relating a universal quantum Turing machine with a local Hamiltonian,
researchers show the problem of telling whether a local Hamiltonian or an initial state thermalize if is uncomputable,
i.e. as hard as the famous halting problem in computer science.
%Another example is recent advances in understanding the hardness of problems from the overlap gap properties that highly inspired from the phase transition in spin glasses.
By relating the energy consumption in computational process and quantum physics,
researchers proved the Landauer's principle from the quantum perspective starting from a simple computational model.

The majority of my current research is about relating these two fields.
One is understanding the solution space properties,
another is embedding computational hard problems to a physical system.

\section{Solution space properties of hard combinatorial optimization problems}

\subsection{Current Work}
\subsection{Future Work}

\section{Algorithms targeting near-term intermediate scale quantum devices}
\subsection{Current Work}
\subsection{Future Work}

\section{Summary}
To summarize, the this point in my career, my primary interests are.

\end{document}
