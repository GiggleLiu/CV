\documentclass[a4paper]{article}
\usepackage{amssymb}
\usepackage{bbm}
\usepackage[normalem]{ulem}
\usepackage{graphicx}
\usepackage[colorlinks,linkcolor=blue,anchorcolor=blue,citecolor=blue,urlcolor=blue]{hyperref}
\usepackage{mciteplus}
\usepackage{etoolbox}
\usepackage{subcaption}
\usepackage{tikz}
\usetikzlibrary{shapes}
\usepackage{fontspec}

\title{Statement of Teaching}
\author{JinGuo Liu\\ Department of Physics, Harvard University}
\date{\today}

\newcommand{\<}{\langle}
\renewcommand{\>}{\rangle}
\newcommand{\vsigma}{\vec{\sigma}}
\setlength{\topmargin}{-10mm}
\setlength{\textwidth}{7in}
\setlength{\oddsidemargin}{-8mm}
\setlength{\textheight}{9in}
\setlength{\footskip}{1in}

\newfontfamily\DejaSans{DejaVu Sans}

\begin{document}
\fontsize{10}{13}
\selectfont
\maketitle

%My first time being a teaching assistant is in the course ``Mathematical Physical Methods'' by Xian-Gang Wan in Nanjing University.
%But this time I did not really have a chance to prepare any lectures.
I only have two short teaching experiences that require me to think about what and how to teach.
One is in the \href{https://for1807.physik.uni-wuerzburg.de/2017/07/11/winter-school-2018/}{The FOR 1807 Winter School on Numerical Methods for Strongly Correlated Quantum Systems}, I and Shuo-Hui Li prepared the \href{https://for1807.physik.uni-wuerzburg.de/2017/07/11/winter-school-2018/}{hands on session}.
The other is in the \href{https://github.com/QuantumBFS/SSSS}{spring school in Song-Shan Lake}, I prepared the lectures on the quantum computing part.
Both are happy and enjoyable experience thanks to my advisor Lei's patient guide.
In both schools, we get good feedbacks. I still remember in the Marburg winter school, Prof. Fakher Assaad so commented to his student in front of Lei: ``See, this is called professional.''.
In the Song-Shan Lake Spring School, we also get a high score in students's feedback.

In the Marburg hands on lecture, we first introduce briefly about the automatic differentiation and high performance computing.
Then we show some examples. After each example, we gave students 10 minutes to play with the online notebook, which they can directly access with a browser without technical barriers.
In the Song-Shan Lake Spring school, I have more flexibility to plot the overall teaching program.
I and other lecturers insist on living in the same apartment with students,
we prepared free food for students at around 8PM so that they can gather and ask any questions they have.
These questions range from setting up develop environment to future research directions.
From our previous experience, most students stuck in the stage of environment setup in a numeric course.
This ``happy fatty night'' aims to solve this issue.

\begin{itemize}
    \item It is important to prepare and archive your teaching material online so that students can access them after the course (see the above links). Github is a good form for organizing things.
    \item If your students have diversed background, it is super helpful to prepare some challenges for them, in case they do not have anything to do if the top students are already quite familiar with the course.
    \item In the begining of each lecture, it is very important to let student know what can they get from this lecture (or the goal) and I need to emphsis this point again and gain.
It can help student to organize what they learnt.
    \item During the teaching, it is important to control the speed.
It is more important to stop at a certain point, and get some feedback from students, than going ``smoothly'' without stop.
This is because, when people encounter a new concept, he needs to think for quite a while in his head before accepting it.
If you did not give him time to savering what he learnt, getting him lost is the only result.
\end{itemize}

I do not have any teaching experience in Harvard partly due to the COVID.
But still I feel these teaching experience helped me a lot in explaining idead to some Phd candidates.

I like teaching. I am primarily interested in teaching ``algorithms''.
It can be in the form of a language course.

\end{document}
