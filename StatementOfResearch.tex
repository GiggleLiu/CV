\documentclass[a4paper]{article}
\usepackage{amssymb}
\usepackage{bbm}
\usepackage[normalem]{ulem}
\usepackage{graphicx}
\usepackage[colorlinks,linkcolor=blue,anchorcolor=blue,citecolor=blue,urlcolor=blue]{hyperref}
\usepackage[numbers]{natbib}
\usepackage{mciteplus}
\usepackage{etoolbox}
\patchcmd{\thebibliography}{\section*{\refname}}{}{}{}
\title{Statement of Research}
\author{JinGuo Liu}
\date{\today}

\newcommand{\<}{\langle}
\renewcommand{\>}{\rangle}
\newcommand{\vsigma}{\vec{\sigma}}
\setlength{\topmargin}{-10mm}
\setlength{\textwidth}{7in}
\setlength{\oddsidemargin}{-8mm}
\setlength{\textheight}{9in}
\setlength{\footskip}{1in}

\begin{document}
\fontsize{10}{13}
\selectfont
\maketitle

\section{Overview of Research Interests}

\section{Previous Work}
\subsection{Tensor Networks and Renormalization Group}
I have two works during my PhD related to tensor networks, one is on the.
Simulating quantum many body systems are my major interests.

\subsection{Quantum Machine Learning}
Quantum machine learning is an exciting field with a lot of chances, I and Lei Wang developed the 

\section{Future Directions}\label{proposal}
\section{Summary}
%\bibliographystyle{ieeetr}
\bibliographystyle{rsc}
\bibliography{ref}

\end{document}
