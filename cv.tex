% LaTeX Curriculum Vitae Template
%
% Copyright (C) 2004-2009 Jason Blevins <jrblevin@sdf.lonestar.org>
% http://jblevins.org/projects/cv-template/
%
% You may use use this document as a template to create your own CV
% and you may redistribute the source code freely. No attribution is
% required in any resulting documents. I do ask that you please leave
% this notice and the above URL in the source code if you choose to
% redistribute this file.

\documentclass[letterpaper]{article}
%\usepackage[UTF8]{ctex}
\usepackage{xeCJK}
\usepackage{hyperref}
\usepackage{geometry}

% Comment the following lines to use the default Computer Modern font
% instead of the Palatino font provided by the mathpazo package.
% Remove the 'osf' bit if you don't like the old style figures.
\usepackage[T1]{fontenc}
\usepackage[sc,osf]{mathpazo}

% Set your name here
\def\name{Jin-Guo Liu (刘金国)}

% Replace this with a link to your CV if you like, or set it empty
% (as in \def\footerlink{}) to remove the link in the footer:
\def\footerlink{}   %{http://jblevins.org/projects/cv-template/}

% The following metadata will show up in the PDF properties
\hypersetup{
  colorlinks = true,
  urlcolor = blue,
  pdfauthor = {\name},
  pdfkeywords = {economics, statistics, mathematics},
  pdftitle = {\name: Curriculum Vitae},
  pdfsubject = {Curriculum Vitae},
  pdfpagemode = UseNone
}

\geometry{
  body={6.5in, 8.5in},
  left=1.0in,
  top=1.25in
}

\fontsize{10}{13}
% Customize page headers
\pagestyle{myheadings}
\markright{\name}
\thispagestyle{empty}

% Custom section fonts
\usepackage{sectsty}
\sectionfont{\rmfamily\mdseries\Large}
\subsectionfont{\rmfamily\mdseries\itshape\large}

% Other possible font commands include:
% \ttfamily for teletype,
% \sffamily for sans serif,
% \bfseries for bold,
% \scshape for small caps,
% \normalsize, \large, \Large, \LARGE sizes.

% Don't indent paragraphs.
\setlength\parindent{0em}

% Make lists without bullets
\renewenvironment{itemize}{
  \begin{list}{}{
    \setlength{\leftmargin}{1.5em}
  }
}{
  \end{list}
}

\begin{document}

% Place name at left
{\huge \name}

% Alternatively, print name centered and bold:
%\centerline{\huge \bf \name}

\vspace{0.2in}

\begin{minipage}{0.45\linewidth}
    \href{http://english.iop.cas.cn/}{Institute of Physics\\
    Chinese Academy of Sciences}\\
    BeiJing 100190, China
\end{minipage}
\begin{minipage}{0.45\linewidth}
    \begin{tabular}{ll}
        Phone: & 86-151-9595-5770 \\
        Email: & \href{mailto:cacate0129@gmail.com}{\tt cacate0129@gmail.com} \\
        Birth: & Jan. 29, 1990.\\
        HomePage: & \href{https://giggleliu.github.io}{https://giggleliu.github.io}\\
        Github: & \href{https://github.com/GiggleLiu}{GiggleLiu}
    \end{tabular}
\end{minipage}



\section*{Education}

\begin{itemize}
    \item B.S. Software Engineering, Nanjing Institution of Science and Technology, 2008--2012.
    \item Ph.D. Physics, Nanjing University, 2012--2017. (Advisor: Prof. Qiang-Hua Wang)
\end{itemize}

\section*{Skills}
\begin{itemize}
    \item Quantum Software Engineering
    \item Tensor Networks
    \item Differentiable Programming
    \item Julia/Python/Fortran language
\end{itemize}

\section*{Awards}
\begin{itemize}
    \item First prize of Physics Olympiad, JiangSu Province, 2007
    \item Academic Excellence Scholarship, Nanjing University, 2016
    \item First prize of \href{http://www.iqiyi.com/w\_19rto3v4h1.html}{ZTE Fantastic Algorithm Challenge} (out of 8000 teams, 100,000 RMB award), 2017
\end{itemize}

\section*{Research interest \& experience}
I am a computational quantum physicsist. Armed with solid background of both quantum physics and computer science, I am able to solve some valuable problems in the cross discipline of quantum physics and computer sciences.
I am also the maintainer of several open source projects (listed at the end of this CV), as well as an organizer of \href{https://github.com/QuantumBFS/}{QuantumBFS}.
I list my research experiences as the following:
\begin{itemize}
    \item [1] When I was a college student, I read a book "Quantum Computation and Quantum Information" by Michael A. Nielsen.
        I was deeply impressed by the beautiful computation framework in the book, and decided to learn more about quantum computing in Prof. Yang Yu's group in Nanjing University.
    \item [2] After one year, I was transferred to Prof. Qiang-Hua Wang's group since I was more interested in theories and numerical simulations rather than experiments. I mastered tensor networks algorithms and renormalizationing group theories, and became a geek in simulating quantum many body systems. Most of my works are about designing numeric algorithms to solve important problems in physics, like multi-channel Kondo problem and fractional topological excitation.
    \item [3] In the last year as a doctor candidate, I won the first prize in ZTE fantastic algorithm challenge, which is a good proof of my solid algorithmic background of matrix computation and combinatorial optimization. Then I became a postdoc of a young and charming guy \href{http://wangleiphy.github.io/}{Lei-Wang}. Besides providing valuable suggestions in my research, Lei also creates a lot of opportunities for me, like encouraging me to give lectures and talks in international conferences and summer schools.
    \item [4] Now is the second year of my postdoc career in Institute of Physics (IOP), Chinese Academy of Sciences. My current research interest is variational quantum algorithms, this is a field that can incubate several killer Apps. I also developed the quantum differentiable learning framework \href{https://github.com/QuantumBFS/Yao.jl}{Yao.jl} together with a genuine Julia lover \href{http://blog.rogerluo.me/}{Xiu-Zhe Luo}.
    \item [5] Also, I am mentoring a student for Julia on the \href{https://summerofcode.withgoogle.com/}{GSoC} project of differentiable tensor networks,
        contracting specific types of tensor networks efficiently is one of the most promising \href{https://arxiv.org/abs/1801.00862}{NISQ} applications in my opinion.
\end{itemize}

\section*{Publications}
\begin{itemize}
    \item [1] {\bf Jin-Guo Liu}, Da Wang and Qiang-Hua Wang, Quantum impurities in channel mixing baths. \href{https://journals.aps.org/prb/abstract/10.1103/PhysRevB.93.035102}{Phys. Rev. B {\bf 93}, 035102} (2016).
    \item [2] {\bf Jin-Guo Liu}, Zhao-Long Gu, Jian-Xin Li and Qiang-Hua Wang, Sub-system fidelity for ground states in one dimensional interacting systems. \href{http://iopscience.iop.org/article/10.1088/1367-2630/aa6a4b}{N. J. Phys. 19(9), 093017} (2017).
    \item [3] Yang Yang, Wan-Sheng Wang, {\bf Jin-Guo Liu}, Hua Chen, Jian-Hui Dai and Qiang-Hua Wang, Superconductivity in doped ${\mathrm{Sr}}_{2}{\mathrm{IrO}}_{4}$: A functional renormalization group study. \href{https://journals.aps.org/prb/abstract/10.1103/PhysRevB.89.094518}{Phys. Rev. B {\bf 89}, 094518} (2014).
    \item [4] Yao Wang, {\bf Jin-Guo Liu}, Wan-Sheng Wang, and Qiang-Hua Wang, Electronic order near the type-II van Hove singularity in BC${}_3$. \href{https://journals.aps.org/prb/abstract/10.1103/PhysRevB.97.174513}{Phys. Rev. B {\bf 97}, 174513} (2018)
    \item [5] Zi Cai, and {\bf Jin-Guo Liu}, Approximating quantum many-body wave functions using artificial neural networks. \href{https://journals.aps.org/prb/abstract/10.1103/PhysRevB.97.035116}{Phys. Rev. B {\bf 97}, 035116} (2018).
    \item [6] {\bf Jin-Guo Liu}, and Lei Wang, Differentiable learning of quantum circuit Born machine. \href{https://arxiv.org/abs/1804.04168}{arXiv:1804.04168} (2018).
    \item [7] Jin-Feng Zeng, Yu-Feng Wu, {\bf Jin-Guo Liu*}, Lei Wang and JiangPing Hu, Learning and Inference on Generative Adversarial Quantum Circuits. \href{https://arxiv.org/abs/1808.03425}{arXiv:1808.03425} (2018)
    \item [8] {\bf Jin-Guo Liu}, Yi-Hong Zhang, Yuan Wan and Lei Wang, Variational Quantum Eigensolver with Fewer Qubits. \href{https://arxiv.org/abs/1902.02663}{arXiv:1902.02663} (2019)
    \item [9] Hai-Jun Liao, {\bf Jin-Guo Liu}, Lei Wang, Tao Xiang, Differentiable Programming Tensor Networks \href{https://arxiv.org/abs/1903.09650}{arXiv:1903.09650} (2019)

\end{itemize}

\section*{A Selection of Github Repositories}
\begin{itemize}
    \item \href{https://github.com/QuantumBFS/Yao.jl}{Yao.jl}: high performance quantum circuit simulator aiming for quantum machine learning. Please also notice
        \begin{itemize}
            \item \href{https://travis-ci.com/QuantumBFS/CuYao.jl}{CuYao.jl}: its GPU extension with orders of \href{https://github.com/QuantumBFS/CuYao.jl/issues/1}{performance improvement} for batched input
            \item \href{https://github.com/QuantumBFS/QuAlgorithmZoo.jl}{QuAlgotithmZoo.jl}, algorithm zoo based on Yao.jl, which includes Grover Search, HHL, QuGAN, QCBM, Hamiltonian Solver et. al.
        \end{itemize}
    \item \href{https://github.com/QuantumBFS/LuxurySparse.jl}{LuxurySparse.jl}: a high performance sparse matrix extension for Julia.
    \item \href{https://github.com/GiggleLiu/marburg}{marbug}: neural network for physicists tutorial code.
    \item \href{https://travis-ci.com/QuantumBFS/FunnyTN.jl}{FunnyTN.jl}: Tensor Network Library for Julia, derived from my old python project \href{https://github.com/GiggleLiu/pymps}{pymps}.
    \item \href{https://github.com/GiggleLiu/viznet}{viznet}: network (neural network, tensor networks and quantum circuit) visualization toolbox.
    \item \href{https://github.com/GiggleLiu/poorman\_nn}{Layers}: computation graph framework with complex value support.
\end{itemize}

\section*{Conferences}
\begin{itemize}
    \item [1] Statistic Physics and Machine Learning (An Qing), talk: "Machine Learning in frustrated quantum spin system".
    \item [2] The FOR 1807 Winter School on Numerical Methods for Strongly Correlated Quantum Systems (Marburg), lecture: "Deep learning and quantum many body systems".
    \item [3] The 8th Workshop on Quantum Many-Body Computation (Hang Zhou), poster: "Differentiable learning of quantum circuit Born machine"
    \item [4] Computational Approaches for Quantum Many Body Systems 2016 (Bei Jing), talk: "Local indistinguishability and topological phase of matter"
    \item [5] The First International Conference on Machine Learning and Physics (Bei Jing), poster: "Differentiable learning of quantum circuit Born machine"
    \item [6] Julia Meetup in BeiJing 2018, talk: Tutorial for high performance matrix computations, in Julia
    \item [7] Quantum Information for Developers 2018 (Zurich), Hackathon: "Funny Tensor Networks"
    \item [8] Song Shan Hu Spring School 2019 (Dong Guan), \href{https://github.com/QuantumBFS/SSSS}{Lecture: "Variational Quantum Computing"}
\end{itemize}

\bigskip

% Footer
\begin{center}
  \begin{footnotesize}
    Last updated: \today \\
    \href{\footerlink}{\texttt{\footerlink}}
  \end{footnotesize}
\end{center}

\end{document}
